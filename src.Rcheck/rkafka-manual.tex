\nonstopmode{}
\documentclass[letterpaper]{book}
\usepackage[times,inconsolata,hyper]{Rd}
\usepackage{makeidx}
\usepackage[utf8,latin1]{inputenc}
% \usepackage{graphicx} % @USE GRAPHICX@
\makeindex{}
\begin{document}
\chapter*{}
\begin{center}
{\textbf{\huge Package `rkafka'}}
\par\bigskip{\large \today}
\end{center}
\begin{description}
\raggedright{}
\item[Type]\AsIs{Package}
\item[Title]\AsIs{R Package for KAFKA}
\item[Version]\AsIs{1.0}
\item[Date]\AsIs{2015-02-11}
\item[Author]\AsIs{Shruti Gupta}
\item[Maintainer]\AsIs{Shruti Gupta}\email{shruti.gupta@mu-sigma.com}\AsIs{}
\item[Description]\AsIs{This package enables the following:1.Creating KAFKA producer 2.Writing messages to a topic 3.Closing KAFKA producer 4.Creating KAFKA consumer 5.Reading messages from a topic 5.Closing KAFKA consumer}
\item[Depends]\AsIs{rJava,RUnit}
\item[SystemRequirements]\AsIs{Oracle Java 7,Apache KAFKA 2.8.0-0.8.1.1}
\item[License]\AsIs{GPL-3}
\end{description}
\Rdcontents{\R{} topics documented:}
\inputencoding{utf8}
\HeaderA{R package for KAFKA}{RKAFKA}{R package for KAFKA}
\aliasA{RKAFKA package}{R package for KAFKA}{RKAFKA package}
\keyword{package}{R package for KAFKA}
%
\begin{Description}\relax
It provides functionalities of creating a KAFKA producer and consumer, and sending and receiving messages.
\end{Description}
%
\begin{Details}\relax

\Tabular{ll}{
Package: & rkafka\\{}
Type: & Package\\{}
Version: & 1.0\\{}
Date: & 2015-02-11\\{}
License: & GPL-3\\{}
}
\textasciitilde{}\textasciitilde{} An overview of how to use the package, including the most important functions \textasciitilde{}\textasciitilde{}
\end{Details}
%
\begin{Author}\relax
Shruti Gupta

Maintainer: Who to complain to shruti.gupta@mu-sigma.com

\end{Author}
%
\begin{References}\relax
\textasciitilde{}\textasciitilde{} Literature or other references for background information \textasciitilde{}\textasciitilde{}
\end{References}
%
\begin{SeeAlso}\relax
\textasciitilde{}\textasciitilde{} Optional links to other man pages, e.g. \textasciitilde{}\textasciitilde{}
\textasciitilde{}\textasciitilde{} \code{\LinkA{<pkg>}{<pkg>}} \textasciitilde{}\textasciitilde{}
\end{SeeAlso}
%
\begin{Examples}
\begin{ExampleCode}
producer1=rkafka.startProducer("127.0.0.1:9092")
rkafka.send(producer1,"ind1","127.0.0.1:9092","this2")
rkafka.send(producer1,"ind1","127.0.0.1:9092","is21")
rkafka.closeProducer(producer1)
consumer1=rkafka.startConsumer("127.0.0.1:2181")
msgs=rkafka.read(consumer1,"ind1")
print(msgs)
rkafka.closeConsumer(consumer1)
\end{ExampleCode}
\end{Examples}
\inputencoding{utf8}
\HeaderA{rkafka.closeConsumer}{Closing KAKFA consumer}{rkafka.closeConsumer}
\keyword{\textbackslash{}textasciitilde{}kafka}{rkafka.closeConsumer}
\keyword{\textbackslash{}textasciitilde{}consumer}{rkafka.closeConsumer}
\keyword{\textbackslash{}textasciitilde{}close}{rkafka.closeConsumer}
%
\begin{Description}\relax
This functions shuts down the KAFKA consumer
\end{Description}
%
\begin{Usage}
\begin{verbatim}
rkafka.closeConsumer(HighConsumerObj)
\end{verbatim}
\end{Usage}
%
\begin{Arguments}
\begin{ldescription}
\item[\code{HighConsumerObj}] 
\#  	  @param HighConsumerObj:Consumer through which messages are to be read(Java Object)	

\end{ldescription}
\end{Arguments}
%
\begin{Value}
Function doesn't return anything
\end{Value}
%
\begin{Author}\relax
Shruti Gupta
\end{Author}
%
\begin{Examples}
\begin{ExampleCode}
consumer1=rkafka.startConsumer("127.0.0.1:2181")
rkafka.closeConsumer(consumer1)
\end{ExampleCode}
\end{Examples}
\inputencoding{utf8}
\HeaderA{rkafka.closeProducer}{KAFKA producer shutdown}{rkafka.closeProducer}
\aliasA{producerObj}{rkafka.closeProducer}{producerObj}
\keyword{\textbackslash{}textasciitilde{}kafka}{rkafka.closeProducer}
\keyword{\textbackslash{}textasciitilde{}producer}{rkafka.closeProducer}
\keyword{\textbackslash{}textasciitilde{}close}{rkafka.closeProducer}
%
\begin{Description}\relax
This function closes the KAFKA producer
\end{Description}
%
\begin{Usage}
\begin{verbatim}
rkafka.closeProducer(producerObj)
\end{verbatim}
\end{Usage}
%
\begin{Arguments}
\begin{ldescription}
\item[\code{producerObj}] 
\#     * @param producerObj:producerObj(Java object)
\#   *            !!Mandatory: Producer which is to be terminated

\end{ldescription}
\end{Arguments}
%
\begin{Value}
Doesn't return anything
\end{Value}
%
\begin{Author}\relax
Shruti Gupta
\end{Author}
%
\begin{Examples}
\begin{ExampleCode}
producer1=rkafka.startProducer("127.0.0.1:9092")
rkafka.closeProducer(producer1)
\end{ExampleCode}
\end{Examples}
\inputencoding{utf8}
\HeaderA{rkafka.read}{KAFKA Consumer Reading}{rkafka.read}
\keyword{\textbackslash{}textasciitilde{}kafka}{rkafka.read}
\keyword{\textbackslash{}textasciitilde{}consumer}{rkafka.read}
\keyword{\textbackslash{}textasciitilde{}read}{rkafka.read}
%
\begin{Description}\relax
This function reads messages received by a KAFKA consumer
\end{Description}
%
\begin{Usage}
\begin{verbatim}
rkafka.read(HighConsumerObj, topicName)
\end{verbatim}
\end{Usage}
%
\begin{Arguments}
\begin{ldescription}
\item[\code{HighConsumerObj}] 
\#  	  @param HighConsumerObj:Consumer through which messages are to be read(Java Object)	

\item[\code{topicName}] 
\#  	  @param topicName
\#*            :The topic from which message is to be read

\end{ldescription}
\end{Arguments}
%
\begin{Value}
Array Of Strings
\end{Value}
%
\begin{Note}\relax
Warning: Ensure to close the consumer after reading messages. Won't work correctly next time otherwise
\end{Note}
%
\begin{Author}\relax
Shruti Gupta
\end{Author}
%
\begin{Examples}
\begin{ExampleCode}
consumer1=rkafka.startConsumer("127.0.0.1:2181")
print(rkafka.read(consumer1,"test"))
\end{ExampleCode}
\end{Examples}
\inputencoding{utf8}
\HeaderA{rkafka.send}{KAFKA producer sending message}{rkafka.send}
\aliasA{ip}{rkafka.send}{ip}
\aliasA{message}{rkafka.send}{message}
\aliasA{producer}{rkafka.send}{producer}
\aliasA{topicName}{rkafka.send}{topicName}
\keyword{\textbackslash{}textasciitilde{}KAFKA}{rkafka.send}
\keyword{\textbackslash{}textasciitilde{}Producer}{rkafka.send}
\keyword{\textbackslash{}textasciitilde{}Message sending}{rkafka.send}
%
\begin{Description}\relax
This function sends message to a particular name through a producer
\end{Description}
%
\begin{Usage}
\begin{verbatim}
rkafka.send(producer, topicName, ip, message)
\end{verbatim}
\end{Usage}
%
\begin{Arguments}
\begin{ldescription}
\item[\code{producer}] 
\#     * @param producer:producer(Java object)
\#   *            !!Mandatory: Producer through which messages are to be sent

\item[\code{topicName}] 
* @param topicName:String
\#   *            !!Mandatory: Topic to which messages are to be sent. If topicName doesn't exist, new topic is created
\#   * 

\item[\code{ip}] 
\#     * @param ip:String
\#   *            !!Mandatory: ip on which producer is running

\item[\code{message}] 
\#     * @param message:String
\#   *            !!Mandatory: message to be sent

\end{ldescription}
\end{Arguments}
%
\begin{Value}
Doesn't return a value
\end{Value}
%
\begin{Author}\relax
Shruti Gupta
\end{Author}
%
\begin{Examples}
\begin{ExampleCode}
producer1=rkafka.startProducer("127.0.0.1:9092")
rkafka.send(producer1,"test","127.0.0.1:9092","Testing")
\end{ExampleCode}
\end{Examples}
\inputencoding{utf8}
\HeaderA{rkafka.startConsumer}{Creating high level KAFKA consumer}{rkafka.startConsumer}
\aliasA{autoCommitEnable}{rkafka.startConsumer}{autoCommitEnable}
\aliasA{autoCommitInterval}{rkafka.startConsumer}{autoCommitInterval}
\aliasA{autoOffsetReset}{rkafka.startConsumer}{autoOffsetReset}
\aliasA{consumerTimeoutMs}{rkafka.startConsumer}{consumerTimeoutMs}
\aliasA{groupId}{rkafka.startConsumer}{groupId}
\aliasA{zookeeperConnect}{rkafka.startConsumer}{zookeeperConnect}
\aliasA{zookeeperConnectionTimeoutMs}{rkafka.startConsumer}{zookeeperConnectionTimeoutMs}
\keyword{\textbackslash{}textasciitilde{}kafka}{rkafka.startConsumer}
\keyword{\textbackslash{}textasciitilde{}consumer}{rkafka.startConsumer}
\keyword{\textbackslash{}textasciitilde{}create}{rkafka.startConsumer}
%
\begin{Description}\relax
This function creates a high level KAFKA consumer
\end{Description}
%
\begin{Usage}
\begin{verbatim}
rkafka.startConsumer(zookeeperConnect,groupId="test-consumer-group",zookeeperConnectionTimeoutMs="100000",consumerTimeoutMs="5000",autoCommitEnable="true",autoCommitInterval="1000",autoOffsetReset="largest")
\end{verbatim}
\end{Usage}
%
\begin{Arguments}
\begin{ldescription}
\item[\code{zookeeperConnect}] 
\#@param zookeeperConnect
\#*            !!Mandatory:Zookeeper connection string comma separated
\#*            host:port pairs, each corresponding to a zk server. e.g.
\#*            "127.0.0.1:3000,127.0.0.1:3001,127.0.0.1:3002"
\#*  		  default:"127.0.0.1:2181"

\item[\code{groupId}] 
\#* @param groupId
\#*            !!Mandatory:consumer group id default:test-consumer-group

\item[\code{zookeeperConnectionTimeoutMs}] 
\#* @param zookeeperConnectionTimeoutMs
\#*            !!Mandatory:timeout in ms for connecting to zookeeper
\#*            default:100000

\item[\code{consumerTimeoutMs}] 
\#* @param consumerTimeoutMs
\#*            !!Mandatory:Throw a timeout exception to the consumer if no
\#*            message is available for consumption after the specified
\#*            interval default:1000

\item[\code{autoCommitEnable}] 
\#*            --Optional:default:true If true, periodically commit to
\#*            ZooKeeper the offset of messages already fetched by the
\#*            consumer. This committed offset will be used when the process
\#*            fails as the position from which the new consumer will begin.

\item[\code{autoCommitInterval}] 
\#* @param autoCommitIntervalMs
\#*            --Optional:default:60*1000 The frequency in ms that the
\#*            consumer offsets are committed to zookeeper.

\item[\code{autoOffsetReset}] 
\#*            --Optional:default:largest * smallest : automatically reset
\#*            the offset to the smallest offset largest : automatically
\#*            reset the offset to the largest offset anything else: throw
\#*            exception to the consumer


\end{ldescription}
\end{Arguments}
%
\begin{Value}
Returns a consumer
\end{Value}
%
\begin{Author}\relax
Shruti Gupta
\end{Author}
%
\begin{Examples}
\begin{ExampleCode}
consumer1=rkafka.startConsumer("127.0.0.1:2181")
consumer2=rkafka.startConsumer("127.0.0.1:2181","test-consumer-group","50000","1000")

\end{ExampleCode}
\end{Examples}
\inputencoding{utf8}
\HeaderA{start.Producer}{Creating producer}{start.Producer}
\aliasA{batchNumMessages}{start.Producer}{batchNumMessages}
\aliasA{compressedTopics}{start.Producer}{compressedTopics}
\aliasA{compressionCodec}{start.Producer}{compressionCodec}
\aliasA{metadataBrokerList}{start.Producer}{metadataBrokerList}
\aliasA{partitionerClass}{start.Producer}{partitionerClass}
\aliasA{producerType}{start.Producer}{producerType}
\aliasA{queueBufferingMaxMessages}{start.Producer}{queueBufferingMaxMessages}
\aliasA{queueBufferingMaxTime}{start.Producer}{queueBufferingMaxTime}
\aliasA{queueEnqueueTimeoutTime}{start.Producer}{queueEnqueueTimeoutTime}
\aliasA{rkafka.startProducer}{start.Producer}{rkafka.startProducer}
\aliasA{serializerClass}{start.Producer}{serializerClass}
\keyword{\textbackslash{}textasciitilde{}KAFKA}{start.Producer}
\keyword{\textbackslash{}textasciitilde{}producer}{start.Producer}
%
\begin{Description}\relax
This function is used to create a KAFKA producer
\end{Description}
%
\begin{Usage}
\begin{verbatim}
rkafka.startProducer(metadataBrokerList,producerType="sync",compressionCodec="none",
  	serializerClass="kafka.serializer.StringEncoder",partitionerClass="NULL",compressedTopics="NULL",
		queueBufferingMaxTime="NULL",
		queueBufferingMaxMessages="NULL",
		queueEnqueueTimeoutTime="NULL",
		batchNumMessages="NULL")
\end{verbatim}
\end{Usage}
%
\begin{Arguments}
\begin{ldescription}
\item[\code{metadataBrokerList}] 
\#     * @param metadataBrokerList:String
\#   *            !!Mandatory list of brokers used for bootstrapping knowledge
\#   *            about the rest of the cluster format: host1:port1,host2:port2
\#   *            ... default:localhost:9092

\item[\code{producerType}] 
\#     * @param producerType:String
\#   *            !!Mandatory specifies whether the messages are sent
\#   *            asynchronously (async) or synchronously (sync) default:sync

\item[\code{compressionCodec}] 
\#     * @param compressionCodec:String
\#   *            !!Mandatory specify the compression codec for all data
\#   *            generated: none , gzip, snappy. default:none

\item[\code{serializerClass}] 
\#     * @param serializerClass:String
\#   *            !!Mandatory message encoder
\#   *            default:kafka.serializer.StringEncoder

\item[\code{partitionerClass}] 
\#     * @param partitionerClass:String
\#   *            --Optional name of the partitioner class for partitioning
\#   *            events; default partition spreads data randomly
\#               default:NULL

\item[\code{compressedTopics}] 
\#     * @param compressedTopics:String
\#   *            --Optional allow topic level compression
\#   * \#               default:NULL

\item[\code{queueBufferingMaxTime}] 
\#     * @param queueBufferingMaxTime:String
\#   *            --Optional(for Async Producer only) maximum time, in
\#   *            milliseconds, for buffering data on the producer queue
\#   * \#               default:NULL

\item[\code{queueBufferingMaxMessages}] 
\#     * @param queueBufferingMaxMessages:String
\#   *            --Optional(for Async Producer only) the maximum size of the
\#   *            blocking queue for buffering on the producer
\#   *\#               default:NULL

\item[\code{queueEnqueueTimeoutTime}] 
\#   *            --Optional(for Async Producer only) 0: events will be enqueued
\#   *            immediately or dropped if the queue is full -ve: enqueue will
\#   *            block indefinitely if the queue is full +ve: enqueue will
\#   *            block up to this many milliseconds if the queue is full
\#   *\#               default:NULL

\item[\code{batchNumMessages}] 
\#     * @param batchNumMessages:String
\#   *            --Optional(for Async Producer only) the number of messages
\#   *            batched at the producer
\#   * \#               default:NULL

\end{ldescription}
\end{Arguments}
%
\begin{Value}
\#     * @return returns a Properties Object containing properties for the
\#   *         Producer, to be passed to MuProducer class
\end{Value}
%
\begin{Author}\relax
Shruti Gupta
\end{Author}
%
\begin{Examples}
\begin{ExampleCode}
producer1=rkafka.startProducer("127.0.0.1:9092")
producer2=rkafka.startProducer("127.0.0.1:9092","sync","none","kafka.serializer.StringEncoder")
  
\end{ExampleCode}
\end{Examples}
\printindex{}
\end{document}
